%%% Filename : ml.tex
%%% Date     : 13.1.1997
%%% Contents : Main file of TeX paper
%%% Author   : Gernot Salzer
%%%
%%% ***
%%% Upgrade  : Wolfram Nix
%%% Date     : ??.11.1997
%%%            preliminaries, lots of definitions, theorems etc added.

\iffalse
\documentstyle[ml,\jobname,proof]{article}  % Headline for the old version of LaTeX (2.09)
\fi
\documentclass{article}
\usepackage{latexsym}
\usepackage{ml}       % General TeX commands for MUltlog
\usepackage{\jobname} % TeX commands specific to the particular logic; written by "ml_tex.pl".
\usepackage{proof}    % Package by Makoto Tatsuta for typesetting Gentzen style proofs

\title{Calculi for the \NameOfLogic\ Logic}
\author{M.~Ultlog}

\begin{document}
\bibliographystyle{plain}
\maketitle

\begin{abstract}
This paper introduces the reader to the 
\NameOfLogic\ logic and to several
related calculi.
\end{abstract}

\section{Introduction}

Within the last years multiple-valued logics, introduced in the 1920's
independently by {\L}ukasiewicz and Post, have attracted considerable
attention by the computer science community due to their potential in the
verification of soft- and hardware, in modelling epistemic states of
knowledge in knowledge base systems etc.~\cite{Haehnle:93}.
The practical success of a logic depends crucially on the availability of
computational calculi for this logic.  Without such calculi, the study of
a logic in relation to its application in computer science and artificial
intelligence is destined to remain purely theoretical. In the case of
many-valued logics, computational calculi can be given.

Analytic calculi for many-valued logics have been known for quite some time.
Sequent systems similar to those presented here have also been introduced by
Schr\"oter~\cite{Schroeter:55}, Rousseau~\cite{Rousseau:67},
Takahashi~\cite{Takahashi:67}, and Carnielli~\cite{Carnielli:91};
equivalent tableaux formulations were given by Surma~\cite{Surma:77} and
Carnielli~\cite{Carnielli:87} (H\"ahnle's work is based on the
latter~\cite{Haehnle:91,Haehnle:92,Haehnle:93}). Many-valued natural deduction
systems have been investigated in Baaz, Ferm\"uller and
Zach~\cite{BaazFermZach93SMVL}. Calculi for automated theorem proving in
many-valued logics are also legion.  Apart from H\"ahnle's work on
tableaux-based theorem proving, various resolution methods have been proposed,
e.g., by O'Hearn and Stachniak~\cite{StachniakOHearn:92} or Baaz and
Ferm\"uller~\cite{BaazFerm92LPAR,BaazFerm95JSC}. For more detailed surveys of
the work done in these areas see~\cite{Haehnle:93,Zach93DA}.

In this paper we present optimized calculi for the \NoTVs-valued
\NameOfLogic\ logic, applying the algorithms developed in
Salzer~\cite{Salzer96CADE,Salzer96KGS}. After defining basic notions and
notations in the next section, we define the \NameOfLogic\ logic in
Section~\ref{sec:matrix}. The following sections are devoted to calculi
for this logic: Section~\ref{sec:sequent} describes a sequent calculus, and
Section~\ref{sec:natural} gives a natural deduction system. The paper
concludes with an outlook on future research topics.

\section{The Syntax of the \NameOfLogic\ Logic}\label{sec:prelim}


%%% Language, Terms, Formulae, Conventions
\begin{definition}

% Giving different defs for propositional logic
\ifprop
   The {\em propositional language~\LL\/} for the \NameOfLogic\ logic
   consists of
   \begin{enumerate}
   \item propositional variables: $x_0, x_1, x_2, \ldots$
   \ifnum\NoOps>0
   \item
         \ifnum\NoOps>1
            propositional connectives,
         \else
            one propositional connective,
         \fi
         arity given in parenthesis:
         $\Opname1 (\Oparity1)$
         \FOR \n=2 \TO \NoOps \DO
            \ifnum\n=\NoOps 
               ,~and~%
            \else
               ,
            \fi 
            $\Opname\n$\ ($\Oparity\n$)%
         \ENDFOR\
   \fi 
   \item auxiliary symbols: ``('', ``)'' and ``,''
   \end{enumerate}

   \ifnum\NoOps=0 
      Every propositional variable and every propositional connective of
      arity 0, called a propositional constant, is a {\em formula}. It is
      called {\em atomic} or an {\em atom}. The \NameOfLogic\ logic only
      contains atomic \formulae.
   \else
      {\em \Formulae\/} are defined inductively:
      \begin{enumerate}
      \item Every propositional variable and every propositional connective of
            arity 0, called a propositional constant, is a formula. It is
            called {\em atomic} or an {\em atom}.
      \item If $A_1,\ldots,A_n$ are \formulae\ and $\operator^n$ is
            a propositional connective of arity~$n$, then
            $\operator^n(A_1,\ldots,A_n)$ is a formula.
      \end{enumerate}
   \fi 
\else
   The {\em first order language~\LL\/} for the \NameOfLogic\ logic
   consists of
   \begin{enumerate}
   \item free variables: $a_0, a_1, a_2, \ldots$
   \item bound variables: $x_0, x_1, x_2, \ldots$
   \item function symbols of arity $i (i \in {\bf N})$, including constants:
         $f_0^i, f_1^i, f_2^i, \ldots$
   \item predicate constants of arity $i (i \in {\bf N})$:
         $P_0^i, P_1^i, P_2^i, \ldots$
   \ifnum\NoOps>0 
   \item
         \ifnum\NoOps>1 
            propositional connectives,
         \else
            one propositional connective,
         \fi 
         arity given in parenthesis:
         $\Opname1 (\Oparity1)$
         \FOR \n=2 \TO \NoOps \DO
            \ifnum\n=\NoOps 
               ~and~%
            \else
               ,
            \fi 
            $\Opname\n$\ ($\Oparity\n$)%
         \ENDFOR\
   \fi 
   \ifnum\NoQus>0 
   \item
         \ifnum\NoQus=1 
            one quantifier:
         \else
            quantifiers:
         \fi 
         $\Quname1$
         \FOR \n=2 \TO \NoQus \DO
            \ifnum\n=\NoQus
               ~and~%
            \else
               ,
            \fi 
            $\Quname\n$
         \ENDFOR\
   \fi 
   \item auxiliary symbols: ``('', ``)'' and ``,''
   \end{enumerate}

   {\em Terms\/} are defined inductively: Every individual constant, free
   or bound variable is a term. If $f^n$ is a function symbol of arity~$n$,
   and $t_1,\dots,t_n$ are terms, then~$f^n(t_1,\ldots,t_n)$ is a term.

   {\em \Formulae\/} are also defined inductively:
   \begin{enumerate}
   \item If $P^n$ is a predicate symbol of arity~$n$, and
         $t_1,\ldots,t_n$ are terms, then $P^n(t_1,\ldots,t_n)$ is
         a formula. It is called {\em atomic} or an {\em atom}.
   \ifnum\NoOps>0 
      \item If $A_1,\ldots,A_n$ are \formulae\ and $\operator^n$ is
            a propositional connective of arity~$n$, then
            $\operator^n(A_1,\ldots,A_n)$ is a formula.
   \fi 
   \item If~$A$ is a formula not containing the bound
         variable~$x$, $a$~is a free variable and $\quantor$ is a quantifier,
         then $(\quantor x)A(x)$, where $A(x)$ is obtained from $A$ by
         replacing $a$ by~$x$ at every occurrence of~$a$ in~$A$, is a formula.
   \end{enumerate}
   A formula is called {\em open}, if it contains free variables, and
   {\em closed} otherwise.  A formula without quantifiers is called {\em
   quantifier-free}.

%The {\em depth\/} of a formula~$A$, or its {\em degree}~$\deg(A)$,
%is inductively defined as follows:
%\[
%   \deg(A) = \Biggl\{ \begin{array}{ll}
%            0 & $if $ A $ is atomic$ \\
%            1 + \sum_{i = 1}^n \deg(A_i)
%            & $if $ A \equiv \operator(A_1,\ldots,A_n) \\
%            \deg(B)+1 & $if $ \equiv (\quantor x)B(x) \end{array}
%\]
\fi % 0 End of different defs for prop and non-prop logic.
\end{definition}


As a notational convention, lowercase letters
\ifprop 
   will be used to denote variables, possibly indexed.
\else
   from the beginning of the alphabet ($a, b, c,\ldots$) will be used
   to denote free variables, $f, g, h, \ldots$ for function symbols and
   constants, $x, y, z, \ldots$ for bound variables,
   all possibly indexed.
\fi 
Uppercase letters $A, B, C, \ldots$ will stand
for \formulae, greek letters $\Gamma, \Delta, \Lambda, \ldots$ for
sequences and sets of
\ifprop 
   \formulae.
\else
   \formulae, $t$ and $s$ for terms.
\fi
\ifnum\NoOps>0 
   \ifnum\NoQus>0
      The symbols $\operator$ and $\quantor$ stand for general propositional
      connectives and quantifiers, respectively.
   \else
      The symbol $\operator$ stands for general propositional connectives.
   \fi
\else
   \ifnum\NoQus>0
      The symbol $\quantor$ stands for general quantifiers.
   \fi
\fi

\ifnotprop
   \begin{definition}
   We will use $\alpha$ as a variable for free variables ({\em eigenvariable})
   and $\tau$ as a variable for terms ({\em term variable}). A formula
   consisting of some formula variables, eigenvariables and term variables is
   called a schema.

   A {\em pre-instance\/} $A'$ of a schema $A$ is an actual formula from the
   \formulae\ of $\LL$ which contains occurrences of the eigenvariables and
   term variables of~$A$.

   An {\em instance\/} $A''$ of $A$ is a pre-instance $A'$ of $A$, where the
   eigenvariables and term variables have been replaced by free variables and
   terms not occurring in $A'$.
%%% Example wanted?
% E.g.\ $A(\alpha,\tau)$ is a schema, $B_1(\alpha,x)\lor B_2(\alpha,\tau,y)$
% is a pre-instance of $A$ and $B_1(z,x)\lor B_2(z,f_1(x),y)$ is an instance.
\end{definition}
\fi


\section{The Semantics of the \NameOfLogic\ Logic}\label{sec:matrix}

%%% Matrix
\begin{definition}
The {\em matrix\/} for the language $\LL$ of the \NameOfLogic\ logic
is given by:
\begin{enumerate}
\item the set of {\em truth values\/}
      $\TVs=\{\TV1\FOR\n=2\TO\NoTVs\DO,\TV\n\ENDFOR\}$,
\item a subset $\TVs^+ \subseteq \TVs$ of {\em designated truth values},
%\item for every propositional connective~\operator\ of arity $i$ and every
%      quantifier~\quantor\ an associated truth function
%      $\toperator \colon \TVs^i \to \TVs$ and
%      $\widetilde{\quantor}\colon \wp(\TVs) \setminus \{\emptyset\} \to \TVs$
%      respectively.
\ifnum\NoOps>0
   \item \ifnum\NoOps=1
            for the propositional connective~$\Opname1$ the
         \else
            for every propositional connective~\operator\ an
         \fi
         associated truth function, as given below.
\fi
\ifnum\NoQus>0
   \item \ifnum\NoQus=1
            for the quantifier~$\Quname1$ the
         \else
            for every quantifier~\quantor\ an
         \fi
         an associated truth function, as given below.
\fi
\end{enumerate}
\end{definition}


%%% ***
%%% Since we are now definitely distinguishing between prop and non-prop
%%% logics the lines for no operators/no quantifiers have been removed
%%% resp. commented out for now.

% ----------- The Operators ------------
\ifnum\NoOps=0
   \ifnum\NoQus=0
%      It is a very simple logic, as it contains neither
%      operators nor quantifiers.
   \else
%      It contains no operators.
   \fi
\else
   The truth function\PS{\NoOps} for connective\PS{\NoOps}
   $\Opname1$%
   \FOR \n=2 \TO \NoOps \DO
      \ifnum\n=\NoOps
         ~and~%
      \else
         ,
      \fi
      $\Opname\n$%
   \ENDFOR\
   \ARE{\NoOps} defined by
   \begin{center}
      $\Optab1$
      \FOR \n=2 \TO \NoOps \DO
         \hfill
         $\Optab\n$
      \ENDFOR
   \end{center}
\fi
% ----------- The Quantifiers ------------
\ifnum\NoQus=0
   \ifnum\NoOps=0
   \else
%      The \NameOfLogic\ logic is purely propositional: it contains no
%      quantifiers.
   \fi
\else
   The truth function\PS{\NoQus} for quantifier\PS{\NoQus}
   $\Quname1$%
   \FOR \n=2 \TO \NoQus \DO
      \ifnum\n=\NoQus
         ~and~%
      \else
         ,
      \fi
      $\Quname\n$%
   \ENDFOR\
   \ARE{\NoQus} defined by
   \begin{center}
      $\Qutab1$
      \FOR \n=2 \TO \NoQus \DO
         \hfill
         $\Qutab\n$
      \ENDFOR
   \end{center}
\fi


%%% Structure, Interpretation, Satisfiability

\ifprop
   \begin{definition}
   Let $A$ be a formula and $x_0, x_1, \ldots, x_k$ the variables occurring
   in $A$. Then an {\em interpretation\/} \I\ of $A$ is an assignment of truth
   values to the variables.
   \end{definition}

   \ifnum\NoOps>0
      \begin{definition}
      Given an interpretation \I, we define the {\em valuation\/} $\val_{\I}$
      for \formulae\ $A$ to truth values as follows:
      \begin{enumerate}
      \item If $A$ is atomic, then $\val_{\I}(A)$ simply is the interpretation
            of $A$.
      \item If $A=\operator(A_1,\ldots,A_n)$,
            where $A_1,\ldots,A_n$ are \formulae, and $\toperator$ is the
            associated truth function to $\operator$, then
            $\val_{\I}(A)=\toperator\bigl(\val_{\I}(A_1),\ldots,\val_{\I}(A_n)\bigr)$.
      \end{enumerate}
      \end{definition}
   \fi
   \begin{definition}
   A formula $A$ is called {\em satisfiable\/} iff there is an
   interpretation~\I\ so that $\val_{\I}(A)\in \TVs^+$, in symbols $\I\models A$,
   and {\em unsatisfiable\/} otherwise.
   $A$ is called a {\em many-valued tautology\/} iff it is
   satisfiable for every interpretation \I.
   \end{definition}
\else
   \begin{definition}
   A {\em structure\/} $\M = \langle D,\Phi_{\M}\rangle$ for
   a language~\LL\ (an~\LL-structure) consists of the following:
   \begin{enumerate}
   \item A nonempty set $D$, called the {\em domain\/} (elements of $D$
         are called {\em individuals}).
   \item A mapping $\Phi_{\M}$ satisfying the following:
         \begin{enumerate}
         \item Each free variable of \LL\ is mapped to an element of $D$.
         \item Each $n$-ary function symbol $f$ of \LL\ is mapped to a
               function $f_{\M}:D^n \to D$, or to an element of $D$ if $n=0$.
               Additionally, $\Phi_{\M}$ maps elements of $D$ to themselves.
         \item Each $n$-ary predicate symbol $P$ of \LL\ is mapped to a
               function $P_{\M}:D^n \to \TVs$, or to a element of $\TVs$ if $n=0$.
         \end{enumerate}
   \end{enumerate}
   \end{definition}

   \begin{definition}
   Let \M\ be an \LL-structure. An {\em assignment\/}~$s$ is a mapping from
   the free variables of \LL to individuals.

   An {\em interpretation} $\I=\langle \M,s\rangle$ is an \LL-structure
   $\M = \langle D,\Phi_{\M}\rangle$ together with an assignment $s$.
   The mapping $\Phi_{\M}$ can be extended in the obvious way to a mapping
   $\Phi_{\I}$ from terms to individuals:
   \begin{enumerate}
   \item If $t$ is a free variable, then $\Phi_{\I}\colon = s(t)$.
   \item If $t$ is of the form $f(t_1,\ldots,t_k)$, where $f$ is a $k$-ary
         function symbol and $t_1,\ldots,t_k$ are terms, then
         $\Phi_{\I} := f_{\M}\bigl(\Phi_{\I}(t_1),\ldots,\Phi_{\I}(t_k)\bigr)$.
   \end{enumerate}
   \end{definition}

   \begin{definition}
   Given an interpretation $\I=\langle \M,s\rangle$, we define the
   {\em valuation\/} $\val_{\I}$ for \formulae\ $A$ to truth values as follows:
   \begin{enumerate}
   \item If $A$ is atomic, $A=P(t_1,\ldots,t_n)$, where $P$ is
         an $n$-ary predicate symbol and $t_1,\ldots,t_n$ are terms, then let
         $\val_{\I}(A)=P_{\M}\bigl(\Phi_{\I}(t_1),\ldots,\Phi_{\I}(t_n)\bigr)$.
   \ifnum\NoOps>0
      \item If $A=\operator(A_1,\ldots,A_n)$,
            where $A_1,\ldots,A_n$ are \formulae, and $\toperator$ is the
            associated truth function to $\operator$, then
            $\val_{\I}(A)=\toperator\bigl(\val_{\I}(A_1),\ldots,\val_{\I}(A_n)\bigr)$.
   \fi
   \item If $A=(\quantor x)(B(x))$, and $\tquantor$ is the
         associated truth function to $\quantor$, then
         $\val_{\I}(A)=\widetilde\quantor\bigl\{\val_{\I}(B(d))|d\in D\}\bigr)$.
   \end{enumerate}
   \end{definition}

   \begin{definition}
   A formula $A$ is called {\em satisfiable\/} iff there is an
   interpretation~\I\ so that $\val_{\I}(A)\in \TVs^+$, in symbols $\I\models A$,
   and {\em unsatisfiable\/} otherwise. $A$ is called a
   {\em many-valued tautology\/} iff it is satisfiable for every
   interpretation \I.
   \end{definition}
\fi

%%% Signed Formulae

\subsection*{Signed \Formulae}

The truth of many-valued \formulae\ can be expressed by means of the
truth of \formulae\ in two-valued logic:
We ``sign'' a formula $A$ with a truth value~$v$ writing it as
its exponent. The intended meaning is: $A^v$ is considered true
if $A$ takes the truth value $v$ and false otherwise.

\begin{definition}
A {\em signed formula\/} is an expression of the form $A^v$, where $A$ is
a many-valued formula and $v$ a truth value. A {\em signed formula
expression\/} (sfe) is a formula built up from signed \formulae\
using~$\sneg$, $\sor$, $\sand$, $\sall$ and $\sex$.

A signed formula expression of the form $A^v$ or $\sneg A^v$ is called a
{\em signed literal}. It is called an {\em atomic literal} iff $A$ is
atomic or {\em signed atom}.
\end{definition}

Thus the logic of signed formula experssions is nothing but classical
logic based on signed \formulae.

% Thus signed formula expressions are Boolean expressions in the signed
% \formulae\ and every interpretation defines a Boolean truth value assignment
% to the signed \formulae\ and hence to the whole expression.

%%% Do you want some more?...
% \begin{definition}
% An sfe $\Gamma$ is called satisfiable iff there is an interpretation so
% that $\Gamma$ is true in the induced valuation. It is called {\em valid\/}
% iff $\Gamma$ is satisfied by every interpretation.
% \end{definition}

% It is easy to see that $\sand$ and $\sor$ are associative, commutative
% and that the distribution law holds. We therefore omit parenthesis with
% the usual conventions: binary operators associate to the right and precedence
% is given in the decreasing order $\sneg,\sand,\sor$.
% If $A_1,\ldots,A_n$ are sfe, let $\bigsand_{i=1}^n A_i$
% ($\bigsor_{i=1}^n A_i$) denote the sfe $A_1\sand \ldots \sand A_n$
% ($A_1\sor \ldots \sor A_n$, respectively).



\section{A Sequent Calculus for the \NameOfLogic\ Logic}\label{sec:sequent}


%%% Syntax & Semantics of Sequents, Sequent Calculus

\begin{definition}[Syntax of Sequents]
A {\em sequent $\Gamma$\/} of {\bf L} is
\ifnum\NoTVs=2
   an ordered pair
\else \ifnum\NoTVs=3
      a triple
\else \ifnum\NoTVs=4
   a quadruple
\else
   a $\NoTVs$-tuple
\fi
\fi
\fi
%%% Idea: If we are dealing with excessively many TVs, we could insert
%%% some dots instead of listing them all.
$\Gamma_{\TV1} \FOR\n=2 \TO \NoTVs \DO | \Gamma_{\TV\n} \ENDFOR$,
of finite sequences~$\Gamma_v$ of \formulae, where $v \in \TVs$.
The $\Gamma_v$ are called the components of $\Gamma$.

For a sequence of \formulae, $\Delta$, and $W \subseteq \TVs$,
let $[W\colon \Delta]$ denote the sequent
whose component $\Gamma_v$ is~$\Delta$ if $v \in W$ and empty otherwise.
For $[\{w_1,\ldots, w_k\}\colon \Delta]$ we also write
$[w_1,\ldots, w_k\colon \Delta]$.
If $\Gamma$ and $\Gamma'$ are sequents, then $\Gamma, \Gamma'$ denotes the
sequent $\Gamma_{\TV1},\Gamma'_{\TV1} | \ifnum\NoTVs>2 \ldots | \fi
\Gamma_{\TV\NoTVs},\Gamma'_{\TV\NoTVs}$.
\end{definition}

\begin{definition}[Semantics of Sequents]
Let \I\ be an interpretation. \I\ {\em satisfies\/} a sequent~$\Gamma$ iff
there is a $v \in \TVs$ so that for some formula $A \in \Gamma_v$,
$\val_\I(F) = v$.  \I\ is called a {\em model\/}
of $\Gamma$, in symbols $\I \models \Gamma$.

$\Gamma$ is called {\em satisfiable\/} iff there is an interpretation \I\
so that $\I \models \Gamma$ and {\em valid\/} iff for every interpretation
\I, $\I \models \Gamma$.
\end{definition}

\begin{definition}
The {\em sequent calculus}~\SC\ for the \NameOfLogic\ is given by:
\begin{enumerate}
\item axiom schemas of the form $[\TVs \colon A]$,  % 1
\item weakening rules for every truth value~$v$:       % 2
      \[
         \infer[$w:$v]{\Gamma, [v \colon A]}{\Gamma}
      \]
\item exchange rules for every truth value~$v$:        % 3
      \[
         \infer[$x:$v]{\Gamma, [v \colon B, A], \Delta}
               {\Gamma, [v \colon A, B], \Delta}
      \]
\item contraction rules for every truth value~$v$:     % 4
      \[
         \infer[$c:$v]{\Gamma, [v \colon A]}{\Gamma, [v \colon A, A]}
      \]
\item cut rules for every two truth values $v \neq w$:        % 5
      \[
         \infer[$cut:$vw]{\Gamma, \Delta}
               {\Gamma, [v \colon A] & \Delta, [w \colon A]}
      \]
\ifnum\NoOps>0
   \item an introduction rule~$\operator \colon v$ for every
         connective~\operator\ and every truth value $v$,
         as specified \ifnum\NoQus>0 below, \else below. \fi  % 6
\fi
\ifnum\NoQus>0
   \item an introduction rule~$\quantor \colon v$ for every
         quantifier~\quantor\ and every truth value $v$,
         as specified below,
         where the free variables~$\alpha$ occurring in the
         upper sequents satisfy the so-called {\em eigenvariable condition}:
         No $\alpha$ occurs in the lower sequent.
\fi
\end{enumerate}                                  % 7
(2)--(5) are called {\em structural rules}.
\ifnum\NoOps>0
   \ifnum\NoQus>0
      (6) and (7) are
   \else
      (6) are
   \fi
   called {\em logical rules}.
\else
   \ifnum\NoQus>0
      (6) are called {\em logical rules}.
   \fi
\fi
\end{definition}


\FOR \op=1 \TO \NoOps \DO
   {\noindent
    The introduction rules for connective $\Opname\op$ are given by
    \begin{center}
       \FOR \tv=1 \TO \NoTVs \DO
          \hspace{2em}%
          \mbox{\infer[\Opname\op{:}\TV\tv]{\Opconcl{\op}{\tv}}{\Opprems{\op}{\tv}}}%
       \ENDFOR
       \hspace{2em}
    \end{center}
   }
\ENDFOR
\FOR \qu=1 \TO \NoQus \DO
   {\noindent
    The introduction rules for quantifier $\Quname\qu$ are given by
    \begin{center}
       \FOR \tv=1 \TO \NoTVs \DO
          \hspace{2em}%
          \mbox{\infer[\Quname\qu{:}\TV\tv]{\Quconcl{\qu}{\tv}}{\Quprems{\qu}{\tv}}}%
       \ENDFOR
       \hspace{2em}%
    \end{center}
   }
\ENDFOR


%%% Proofs, Theorems

\begin{definition}[Proof]
An upward tree~$P$ of sequents is called a {\em proof} in the sequent
calculus~\SC\ iff every leaf is
\ifnotprop an instance of \fi an axiom in~\SC, and
all other sequents in it are obtained from the ones standing
immediately above it by applying one of the rules of~\SC.
The sequent at the root of~$P$ is called its {\em end-sequent}. A
sequent~$\Gamma$ is called {\em provable} in \SC, in symbols:
$\vdash^\SC \Gamma$ iff it is the end-sequent of some proof in~\SC.
\end{definition}

\begin{theorem}[Soundness and Completeness for \SC\ of the \NameOfLogic\ logic]
A sequent is provable, if and only if it is valid.
\end{theorem}

For a proof of these theorems see Zach~\cite{Zach93DA}.




\section{Natural Deduction}\label{sec:natural}


%%% Natural Deduction Systems, Formulation

Let $\Gamma$ be a (set) sequent, $\TVs^+ \subseteq \TVs$ the set of
{\em designated truth values}.  The set of non-designated truth values is then
$\TVs^- = \TVs\setminus \TVs^+$.  We divide the sequent $\Gamma$ into its
designated part $\Gamma^+$ and its non-designated part $\Gamma^-$ in
the obvious way:
\begin{eqnarray*}
   \Gamma^+ & := & \langle \Gamma_v \mid v \in \TVs^+ \rangle \\
   \Gamma^- & := & \langle \Gamma_v \mid v \in \TVs^- \rangle
\end{eqnarray*}

\begin{definition}
The {\em natural deduction system} for the \NameOfLogic\ logic is given by:
\begin{enumerate}
\item Assumptions of the form $[\TVs^- \colon A]$ where $A$ is any formula.
\item A weakening rule for all $v \in \TVs^+$:
      \[
         \infer[$w$\colon v]{\Gamma^+, [v \colon A]}{\deduce{\Gamma^+}{\Gamma^-}}
      \]
Weakenings are considered as introduction rules.
\ifnum\NoOps>0
   \item For every connective $\Box$ and every truth value $v$ an
         introduction rule $\Box$:I$_v$ (if $v \in \TVs^+$) or an elimination
         rule $\Box$:E$_v$ (if $v \in \TVs^-$).
\fi
\ifnum\NoQus>0
   \item For every quantifier \quantor\ and every truth value $v_i$ an
         introduction rule \quantor:I$_i$ (if $v_i \in \TVs^+$) or an
         elimination rule~\quantor:E$_i$ (if $v_i \in \TVs^-$).
\fi
\end{enumerate}
\end{definition}




\FOR \op=1 \TO \NoOps \DO
   {\noindent
    The introduction and elimination rules for connective $\Opname\op$ are given by
    \begin{center}
       \FOR \tv=1 \TO \NoTVs \DO
          \hspace{2em}%
          \mbox{\infer[\Opname\op{:}\NDEI\tv_{\TV\tv}]{\NDOpconcl{\op}{\tv}}{\NDOpprems{\op}{\tv}}}%
       \ENDFOR
       \hspace{2em}%
    \end{center}
   }
\ENDFOR
\FOR \qu=1 \TO \NoQus \DO
   {\noindent
    The introduction and elimination rules for quantifier $\Quname\qu$ are given by
    \begin{center}
       \FOR \tv=1 \TO \NoTVs \DO
          \hspace{2em}%
          \mbox{\infer[\Quname\qu{:}\NDEI\tv_{\TV\tv}]{\NDQuconcl{\qu}{\tv}}{\NDQuprems{\qu}{\tv}}}%
       \ENDFOR
       \hspace{2em}%
    \end{center}
   }
\ENDFOR



%%% Natural Deduction Systems, Misc.

\begin{definition}
A {\em natural deduction derivation} is defined inductively as follows:
\begin{enumerate}
\item Let $A$ be any formula. Then
      \[
         \infer{ [\TVs^+ \colon A] }{ [\TVs^- \colon A]}
      \]
      is a derivation of $A$ from the assumption $[\TVs^- \colon A]$
      (an {\em initial derivation}).
\item If $D_k$ are derivations of $\Gamma_k^+, \Delta_k^+$ from the
      assumptions $\Gamma_k^-, \hat{\Delta}_k^-$, and
      \[
         \infer{\Pi^+}
               {\left\langle
               \begin{array}{c}
                  \Gamma_k^-, \lceil \Delta_k^- \rceil \\
                  \Gamma_k^+, \Delta_k^+
               \end{array}
               \right\rangle_{k \in K}}
      \]
      is an instance of a deduction rule with $\hat{\Delta}_k^-$ a
      subsequent of $\Delta_k^-$,
      and all eigenvariable conditions are satisfied, then
      \[
         \infer{\Pi^+}{\langle D_k \rangle_{k \in K}}
      \]
      is a derivation of $\Pi^+$ from the assumptions
      $\bigcup_{k \in K}\Gamma_k^-$.
      The \formulae\ in $\hat{\Delta}_k^-$ which do not occur in
      $\bigcup_{k \in K} \Gamma_k^-$
      are said to be {\em discharged} at this inference.
\end{enumerate}
\end{definition}


\begin{definition}
We call a formula occurrence $A$
\begin{enumerate}
\item the {\em conclusion formula} of an introduction, if it is the formula
      being introduced, i.e., it is $F$ in the conclusion $[i \colon F]$;
\item a {\em premise formula} of an introduction, if it is one of the \formulae\
      in $\Delta'_{f\colon i}(j)^+$ in that introduction;
\item a {\em major premise formula} of an elimination, if it is among the
      formula being eliminated, i.e., in the major premise $[\TVs^+ \colon F]$;
\item a {\em minor premise formula} of an elimination, if it is among
      the \formulae\ in $\Delta'_{f\colon i}(j)^+$ in that elimination,
\item a {\em discharged assumption formula} of an elimination, if
      it stands immediately below an assumption which contains
      the \formulae\ in $\Delta'_{f\colon i}(j)^-$ being discharged
      at that elimination.
\end{enumerate}
A formula occurrence $A$ is said to {\em follow} $A'$, if both
are of the same form and $A'$ stands immediately above $A$
at the same position.
\end{definition}

\begin{theorem}[Soundness]
If a partial sequent $\Gamma^+$ can be derived from the
assumptions $\Gamma^-$, then the following holds for every
interpretation~$\I$:
If no formula in $\Gamma^-_v$ ($v \in \TVs^-$) evaluates
to the truth value $v$, then there is a $w \in \TVs^+$ and a formula
in $\Gamma^+_w$ that evaluates to $w$.
\end{theorem}

\begin{theorem}[Completeness]
Natural deduction systems are complete.
\end{theorem}

For a proof see Zach~\cite{Zach93DA}.



\section{Clause Formation Rules}\label{sec:clauses}


%%% Clause Formation Calculi, Clauses, Semantics of Clauses

\begin{definition}
A {\em (many-valued) clause~$C = \{ A_1^{w_1}, \ldots, A_n^{w_n}\}$} is a
finite set of signed atoms (proper clause).
The empty clause is denoted by $\Box$.
% The {\em atom set~at$(C)$} of a clause $C$ is
% the set of its atomic \formulae:
% at$(C) = \{A_1, \ldots, A_n\}$.

An {\em extended clause} is a finite set of signed \formulae.
\end{definition}

\begin{definition}[Semantics of Clauses and Sets of Clauses]
We translate clauses (clause sets) into signed formula expression by
defining:
\begin{eqnarray*}
   \mbox{sfe}(\Clause) \defeq \sall \vec{x} \Sor_{L \in \Clause} L \\
   \mbox{sfe}({\ClauseSet}) \defeq \Sand_{\Clause \in \ClauseSet}
   \mbox{sfe}(\Clause)
\end{eqnarray*}
Satisfiability and validity can then be expressed in terms of sfes:
A clause $\Clause$ (clause set $\ClauseSet$) is satisfiable (valid),
if and only if sfe$(\Clause)$ (sfe$(\ClauseSet)$) is satisfiable (valid).
\end{definition}

%%% I assume that the general definition for the translation rules
%%% are not required as they weren't for int rules at SC and ND.
% \begin{definition}
% Let $\bigsand_{j \in I}\Delta_{\operator\colon i}(j)$ be an
% $i$-th~partial normal form of $\operator(A_1, \ldots, A_n)$, where $\operator$
% is of arity~$n$, $\Delta_{\operator\colon i}(j)$ is a disjunction of signed
% atoms of the form $A_i^w$ ($1 \le i \le n$), and let
% $\Delta'_{\operator\colon i}(j)$ be the clause corresponding
% to~$\Delta_{\operator\colon i}(j)$
%
% A {\em translation rule for~$\operator$ at place~$i$}
% is a schema of the form:
% \[
 %   \infer[\operator$:$i]
%          {\C \cup \bigl\{
%          C \cup \Delta'_{\operator\colon i}(j)
%          \mid j \in I\}\bigr\}}
%          {\C \cup \bigl\{C \cup \{\operator(A_1,\ldots, A_n)^{v_i}\}\bigr\}}
% \]
% \end{definition}
%
% \begin{definition}
% Let $\bigsand_{j \in I}\Delta_{\quantor\colon i}(j)$ be an
% $i$-th partial normal form of $(\quantor x)A(x)$, where
% $\Delta_{\quantor\colon i}(j)$ is a disjunction of signed
% atoms of the form $A(\tau_i)^w$ ($1 \le i \le p$) respectively
% $A(\alpha_i)^w$ ($1 \le i \le q$), and let
% $\Delta'_{\quantor\colon i}(j)$ be the clause corresponding
% to~$\Delta_{\quantor\colon i}(j)$.
%
% A {\em translation rule for~$\quantor$ at place~$i$}
% is a schema of the form:
% \[
%    \infer[\operator \colon i]
%          {\D = \C \cup \bigl\{
%          C \cup \Delta''_{\quantor\colon i}(j)
%          \mid j \in I\}\bigr\}}
%          {\E = \C \cup \bigl\{C \cup
%          \{ \bigl((\quantor x)A(x)\bigr)^{v_i} \}\bigr\}}
% \]
% where $\Delta''_{\quantor\colon i}(j)$ is obtained
% from $\Delta'_{\quantor\colon i}(j)$ by
% \begin{enumerate}
% \item replacing term variables $\tau_i$ by terms of the form
%       $f_i(a_1, \ldots, a_k)$, where $f_i$ are distinct new $k$-ary
%       function symbols and $a_1$, \dots, $a_k$ are all free variables in
%       $C \cup \{ \bigl((\quantor x)A(x)\bigr)^{v_i}\}$, and by
% \item replacing eigenvariables $\alpha_i$ by distinct new free
%       variables $b_i$ (not occurring in \E).
% \end{enumerate}
% \end{definition}


\FOR \op=1 \TO \NoOps \DO
   {\noindent
    The clause formation rules for connective $\Opname\op$ are given by
    \begin{center}
       \FOR \tv=1 \TO \NoTVs \DO
          \hspace{2em}%
          \mbox{\infer[\Opname\op{:}\TV\tv]%
                      {\ClauseSet\ClOpconcl{\op}{\tv}}%
                      {\ClauseSet\ClOpprem {\op}{\tv}}%
               }%
       \ENDFOR
       \hspace{2em}
    \end{center}
   }
\ENDFOR
\FOR \qu=1 \TO \NoQus \DO
   {\noindent
    The clause formation rules for quantifier $\Quname\qu$ are given by
    \begin{center}
       \FOR \tv=1 \TO \NoTVs \DO
          \hspace{2em}%
          \mbox{\infer[\Quname\qu{:}\TV\tv]%
                      {\ClauseSet\ClQuconcl{\qu}{\tv}}%
                      {\ClauseSet\ClQuprem {\qu}{\tv}}%
               }%
       \ENDFOR
       \hspace{2em}%
    \end{center}
   }
\ENDFOR


%%% Clause Formation Calculi, Soundness, Completeness

% I am assuming here that the CFC-chapter is entirely skipped, if
% neither connectives nor quantifiers are defined.
\begin{theorem}
Let {\cal D} be the result of exhaustively applying the translation rules
to a clause set $\ClauseSet$. Then {\cal D} is a set of proper clauses, i.e.\
{\cal D} contains only signed atoms%
\ifnum\NoOps>0%
   \ifnum\NoQus>0%
      \ (all many-valued connectives and quantifiers are eliminated)%
   \else
      \ (all many-valued connectives are eliminated)%
   \fi
\else
   \ifnum\NoQus>0%
      \ (all many-valued quantifiers are eliminated)%
   \fi
\fi
. Furthermore, {\cal D}~is equivalent to $\ClauseSet$ with respect to
satisfiability.
\end{theorem}

\section{Future directions of research}

In this paper we presented an optimal sequent calculus, an optimal natural
deduction system, and optimal clause formation rules.
Currently we are searching
for optimized versions of other
calculi for the \NameOfLogic\ logic, like tableau systems or 
negative variants of sequent calculus.
Another interesting topic are calculi based
on sets as signs~\cite{Haehnle:93}.
On the practical side, we intend to implement the calculi in a prototype using
Prolog.

\section*{Acknowledgements}

I am deeply indebted to my creators. In particular I would like to thank
M.~Baaz, C.~Ferm\"uller and R.~Zach for laying the foundations, and G.~Salzer
for calling me into existence. Last but not least, I am grateful to
A.~Leitgeb, W.~Nix and M.~Schranz for their cosmetical surgery,
greatly enhancing my outward appearance.

\bibliography{ml}

\end{document}

